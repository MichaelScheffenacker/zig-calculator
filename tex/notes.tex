\documentclass[a4paper,10pt]{article}
\usepackage[utf8]{inputenc}
% \usepackage{amsmath,mathtools}

%opening
\title{Calculator Notes}
\author{msc}

\begin{document}

% \maketitle

% \begin{abstract}

% \end{abstract}

% \section{}

\begin{verbatim}
 $ 1 / 2
 1
 -
 2
\end{verbatim}

$$\frac{1}{2}$$


\begin{verbatim}
 $ 1 / 2 / 3
  1
  -
  2
 ---
  3
\end{verbatim}

\[\frac{\frac{1}{2}}{3}\]


\begin{verbatim}
 $ 1 / 2 / 3 / 4
   1
   -
   2
  ---
   3
 -----
   4
\end{verbatim}
\[
\frac{\frac{\frac{1}{2}}{3}}{4}=\frac{\frac{1}{2\cdot3}}{4} = \frac{1}{2\cdot3\cdot4}
\]


\begin{verbatim}
 $ 1 / 2 / (3 / 4)
\end{verbatim}
\[
\frac{\frac{1}{2}}{\frac{3}{4}}=\frac{1\cdot4}{2\cdot3}
\]


\begin{verbatim}
 $ 5 * 6 * 1 / 2 / 3 / 4
\end{verbatim}
\[
5\cdot6\cdot\frac{\frac{\frac{1}{2}}{3}}{4} =
5\cdot6\cdot\frac{\frac{1}{2\cdot3}}{4} =
5\cdot6\cdot\frac{1}{2\cdot3\cdot4} =
30\cdot\frac{1}{24} =
\frac{30}{24} =
\frac{15}{12} =
\frac{5}{4}
\]


\begin{verbatim}
 $ 1 / 2 / 3 / 4 * 5 * 6
\end{verbatim}
\[
\frac{\frac{\frac{1}{2}}{3}}{4}\cdot5\cdot6 =
\frac{\frac{1}{2\cdot3}}{4}\cdot5\cdot6 =
\frac{1}{2\cdot3\cdot4}\cdot5\cdot6 =
\frac{1}{24}\cdot30 =
\frac{30}{24} =
\frac{15}{12} =
\frac{5}{4}
\]


\begin{verbatim}
 $ 1 / 2 * 5 / 3 * 6 / 4
\end{verbatim}
\[
\frac{1}{2}\cdot\frac{5}{3}\cdot\frac{6}{4} =
\frac{1}{2}\cdot\frac{5}{3}\cdot\frac{3}{2} =
\frac{1\cdot5}{2\cdot3}\cdot\frac{6}{4} =
\frac{5}{6}\cdot\frac{6}{4} =
\frac{5\cdot6}{6\cdot4} =
\frac{30}{24} =
\frac{5}{4}
\]


\begin{verbatim}
 $ 1 / 2 / 3 / 4 + 5 / 6
   1
   -
   2
  ---
   3     5
 ----- + -
   4     6
\end{verbatim}


\begin{verbatim}
 $ 1 / 2
   1
   -
   2
\end{verbatim}


\begin{verbatim}
 $ 1 / 2 / 3 / 4
     1
   -----
   2·3·4
\end{verbatim}


\begin{verbatim}
 $ 1 / 2 / 3 / 4 * 5 * 6
   1·5·6
   -----
   2·3·4
\end{verbatim}



\section{Prime Factorization}

Brute force calculation of primes failed;
even the first $10^5$ take 10\,s here on Witt\-gen\-stein.

Proper prime factorization algorithms that factorize numbers above $10^{100}$ would at least require to look into number theory and algebra.
Apparently implementation of such algorithms took something in the order of years rather than months.

A better idea might be to implement a prime sieve to calculate a sufficient number of primes and then brute force divide through the numbers.


\subsection{Sieve of Eratosthenes}
This is the simplest sieve which just removes all multiples of each prime number (within the array).
A simple and efficient lookup might be an array with all numbers up to the upper bound.
A reasonable upper bound might be $2^{33} = 8\,589\,934\,592$ .
This should safely allow to factorize all numbers up to $(2^{33})^2 = 2^{66} = 73\,786\,976\,294\,838\,206\,464$
[better check by investigating the largest primes smaller than this numbers].

Utilizing an array with the full range of numbers allows to identify the offset as the value of the number,
therefore the array's value can be used as boolean prime indicator.

\begin{verbatim}
0 1 2 3 4 5 6 7 8 9
0 0 1 1 1 1 1 1 1 1
    ^ first prime: 2

0 1 2 3 4 5 6 7 8 9
0 0 1 1 0 1 0 1 0 1
        ^   ^   ^ multiples of 2

0 1 2 3 4 5 6 7 8 9
0 0 1 1 0 1 0 1 0 1
      ^ prime: 3

0 1 2 3 4 5 6 7 8 9 0 1 2 3 4 5
0 0 1 1 0 1 0 1 0 0 0 1 0 1 0 0
            ^     ^     ^     ^ multiples of 3

0 1 2 3 4 5 6 7 8 9 0 1 2 3 4 5
0 0 1 1 0 1 0 1 0 0 0 1 0 1 0 0
          ^ prime: 5

...
\end{verbatim}

[apparently the smallest untouched multiple of a new prime is its square, might want to include that ...]






\end{document}
