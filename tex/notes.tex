\documentclass[a4paper,10pt]{article}
\usepackage[utf8]{inputenc}
\usepackage{amsfonts}
\usepackage{amsmath}
\usepackage{mathtools}
\usepackage{soul}
\usepackage[style=iso]{datetime2}

%opening
\title{Calculator Notes}
\author{msc}

\begin{document}

% \maketitle

% \begin{abstract}

% \end{abstract}


\section{Input Rendering}

\begin{verbatim}
 $ 1 / 2
 1
 -
 2
\end{verbatim}

$$\frac{1}{2}$$


\begin{verbatim}
 $ 1 / 2 / 3
  1
  -
  2
 ---
  3
\end{verbatim}

\[\frac{\frac{1}{2}}{3}\]


\begin{verbatim}
 $ 1 / 2 / 3 / 4
   1
   -
   2
  ---
   3
 -----
   4
\end{verbatim}
\[
\frac{\frac{\frac{1}{2}}{3}}{4}=\frac{\frac{1}{2\cdot3}}{4} = \frac{1}{2\cdot3\cdot4}
\]


\begin{verbatim}
 $ 1 / 2 / (3 / 4)
\end{verbatim}
\[
\frac{\frac{1}{2}}{\frac{3}{4}}=\frac{1\cdot4}{2\cdot3}
\]


\begin{verbatim}
 $ 5 * 6 * 1 / 2 / 3 / 4
\end{verbatim}
\[
5\cdot6\cdot\frac{\frac{\frac{1}{2}}{3}}{4} =
5\cdot6\cdot\frac{\frac{1}{2\cdot3}}{4} =
5\cdot6\cdot\frac{1}{2\cdot3\cdot4} =
30\cdot\frac{1}{24} =
\frac{30}{24} =
\frac{15}{12} =
\frac{5}{4}
\]


\begin{verbatim}
 $ 1 / 2 / 3 / 4 * 5 * 6
\end{verbatim}
\[
\frac{\frac{\frac{1}{2}}{3}}{4}\cdot5\cdot6 =
\frac{\frac{1}{2\cdot3}}{4}\cdot5\cdot6 =
\frac{1}{2\cdot3\cdot4}\cdot5\cdot6 =
\frac{1}{24}\cdot30 =
\frac{30}{24} =
\frac{15}{12} =
\frac{5}{4}
\]


\begin{verbatim}
 $ 1 / 2 * 5 / 3 * 6 / 4
\end{verbatim}
\[
\frac{1}{2}\cdot\frac{5}{3}\cdot\frac{6}{4} =
\frac{1}{2}\cdot\frac{5}{3}\cdot\frac{3}{2} =
\frac{1\cdot5}{2\cdot3}\cdot\frac{6}{4} =
\frac{5}{6}\cdot\frac{6}{4} =
\frac{5\cdot6}{6\cdot4} =
\frac{30}{24} =
\frac{5}{4}
\]


\begin{verbatim}
 $ 1 / 2 / 3 / 4 + 5 / 6
   1
   -
   2
  ---
   3     5
 ----- + -
   4     6
\end{verbatim}


\begin{verbatim}
 $ 1 / 2
   1
   -
   2
\end{verbatim}


\begin{verbatim}
 $ 1 / 2 / 3 / 4
     1
   -----
   2·3·4
\end{verbatim}


\begin{verbatim}
 $ 1 / 2 / 3 / 4 * 5 * 6
   1·5·6
   -----
   2·3·4
\end{verbatim}



\section{Prime Factorization}

Brute force calculation of primes failed;
even the first $10^5$ take 10\,s here on Witt\-gen\-stein.

Proper prime factorization algorithms that factorize numbers above $10^{100}$ would at least require to look into number theory and algebra.
Apparently implementation of such algorithms took something in the order of years rather than months.

A better idea might be to implement a prime sieve to calculate a sufficient number of primes and then brute force divide through the numbers.


\subsection{Sieve of Eratosthenes}
This is the simplest sieve which just removes all multiples of each prime number (within the array).
A simple and efficient lookup might be an array with all numbers up to the upper bound.
A reasonable upper bound might be $2^{33} = 8\,589\,934\,592$ .
This should safely allow to factorize all numbers up to $(2^{33})^2 = 2^{66} = 73\,786\,976\,294\,838\,206\,464$
[better check by investigating the largest primes smaller than this numbers].

Utilizing an array with the full range of numbers allows to identify the offset as the value of the number,
therefore the array's value can be used as boolean prime indicator.

\begin{verbatim}
0 1 2 3 4 5 6 7 8 9
0 0 1 1 1 1 1 1 1 1
    ^ first prime: 2

0 1 2 3 4 5 6 7 8 9
0 0 1 1 0 1 0 1 0 1
        ^   ^   ^ multiples of 2

0 1 2 3 4 5 6 7 8 9
0 0 1 1 0 1 0 1 0 1
      ^ prime: 3

0 1 2 3 4 5 6 7 8 9 0 1 2 3 4 5
0 0 1 1 0 1 0 1 0 0 0 1 0 1 0 0
            ^     ^     ^     ^ multiples of 3

0 1 2 3 4 5 6 7 8 9 0 1 2 3 4 5
0 0 1 1 0 1 0 1 0 0 0 1 0 1 0 0
          ^ prime: 5

...
\end{verbatim}

[apparently the smallest untouched multiple of a new prime is its square, might want to include that ...]


\subsection{Numerical Limits}

\begin{itemize}
 \item Condensate eight Booleans into a byte; would reduce memory usage to an eighth. To evaluate the overhead of extracting the values would require testing.
 \begin{itemize}
  \item Lower numbers would require to bitwise-and their primality out of the byte.
  \item In the range of larger numbers, prime numbers get more sparse.
  The last prime number before $2^{33}$ is 8\,589\,934\,583.
  According to Wolfram Alpha 8\,589\,934\,583 is the 393\,615\,806\textsuperscript{th} prime number.
  This means on average up to this prime number about every 22\textsuperscript{nd} number is prime.
  So even if the prime numbers were evenly spread over this range almost two out of three bytes would be zero,
  allowing a simple, initial zero check before having to bitwise-and the number.
  (If this this actually more efficient is questionable since the bitwise and should be cheap anyways
  while transferring the data from memory might be the expensive operation.)
  \item Another idea:
  If most of the bytes are zero anyways there might be a relatively simple function that identifies the offsets of most of the zero bytes.
  In this case running this function from cache might be faster than fetching all this data from main memory.
  On the other hand the prefetcher might nullify this advantage.
 \end{itemize}
 \item Skip every second value.
 Since, except for two, even numbers are not prime anyways,
 skipping every even number should half calculation time and memory space.

\end{itemize}


\marginpar{2024-09-16}

Numbers up to $2^{33}$ obviously requires 64 bit unsigned integers;
to save space it might be sufficient to utilize prime numbers up to $2^{32}$.
The largest prime number smaller than $2^{32}$ is $2^{32} - 5 = 4\,294\,967\,291$
which is the 203\,280\,221\textsuperscript{st} prime number.
Therefore an array holding all primes numbers up to $2^{32} - 5$ as unsigned 32 bit integers would require around 800\,MB.
The closest prime number below $(2^{32} - 5)^2$ is $(2^{32} - 5)^2 - 2 = 18\,446\,744\,030\,759\,878\,679$.
But integers greater than this number can still be factorized by utilizing $2^{32} - 5$
since the division of $(2^{32} - 5)^2 - 2$ will fail
and therefore render it prime.
But for larger integers divisions will succeed
since they are not prime and therefore prime numbers up to $2^{32} - 5$ will factorize them.
The first number this will fail is the first prime number larger than $(2^{32} - 5)^2$
which is $(2^{32} - 5)^2 + 40 = 18\,446\,744\,030\,759\,878\,721$.
Therefore every integer up to 18\,446\,744\,030\,759\,878\,720 can be factorized by the set of all prime numbers up to $2^{32} - 5$.

\marginpar{2024-09-19}
Okay, we need some nomenclature here:
\begin{itemize}
 \item $B_{32} := 2^{32} = 4\,294\,967\,296$ one above the largest \texttt{unsigned integer}
 \item $B^-_{32} := B_{32} - 5 = 2^{32} - 5 = 4\,294\,967\,291$ the largest prime number smaller than $B_{32}$ (203\,280\,221\textsuperscript{st} prime number)
 \item $B^+_{32} := B_{32} + 15 = 2^{32} + 15 = 4\,294\,967\,311$ the smallest prime number larger than $B_{32}$ (203\,280\,222\textsuperscript{nd} prime number)
 \item $B_{\dot{64}} := {B^-_{32}}^2 = (2^{32} - 5)^2$
 \item $B^-_{\dot{64}} := B_{\dot{64}} - 2 = (2^{32} - 5)^2 - 2 = 18\,446\,744\,030\,759\,878\,679$
 \item $B^+_{\dot{64}} := B_{\dot{64}} + 40 = (2^{32} - 5)^2 + 40 = 18\,446\,744\,030\,759\,878\,721$
\end{itemize}


There is an error in the considerations above:
Primes numbers larger than $B_{\dot{64}}$ can actually be detected
since finding no prime divisor up to $B^-_{32}$ simply means the number in question is prime;
this is true up to the smallest number that has two prime factors larger than $B^-_{32}$.
(For example this is not true for $B^-_{32}\cdot B^+_{32}$
since $B^-_{32}$ is in the \texttt{prime number array}
and after the division the result is $B^+_{32}$ which turns out to be prime and the factorization is complete.)
The smallest number composed of the smallest two prime numbers larger than $B^-_{32}$ is ${B^+_{32}}^2 = 18\,446\,744\,202\,558\,570\,721$.
${B^+_{32}}^2$ is 171\,798\,692\,000 larger than $B^+_{\dot{64}}$,
the former candidate for the upper limit of the prime factorization with factors up to $B^-_{32}$.

\marginpar{2024-09-23}

\newcommand{\PP}{\mathbb{P}}
\newcommand{\NN}{\mathbb{N}}
\newcommand{\TT}{\mathbb{T}}
\newcommand{\Bm}[1]{B^-_{32}}
\newcommand{\Bp}[1]{B^+_{32}}
\newcommand{\set}[1]{\lbrace \, {#1} \, \rbrace}
\newcommand{\Set}[2]{\set{ {#1} \;|\; {#2} }}

Let's try to bring this into a more formal form.
$\PP$ the Set of all prime numbers.

$$\PP_{\Bm{32}} = \Set{p}{p \in \PP \leq \Bm{32}}$$

$$\PP_{\Bp{32}} = \Set{p}{p \in \PP \leq \Bp{32}} = \set{2, 3, 5, \dots, \Bm{32}, \Bp{32}}$$

$$\PP_{x} \coloneq \Set{p}{p \in \PP \leq x \in \PP}$$

$n \coloneq |\PP_x|$; $p_i \in \PP$

$$\PP_{p_n}
= \set{2, 3, 5, \dots, p_n}
= \set{p_1, p_2, p_3, \dots, p_{n-1}, p_n}$$

An integer $x \in \NN \geq 2$ is called factorizable by $\PP_{p_n}$
if either all prime factors $p_i$ of $x$ are in $\PP_{p_n}$
or if by utilizing $\PP_{p_n}$ it can be guaranteed that $x$ is prime itself
and therefore its only prime factor.
Any $x$ is factorizable by $\PP$ since by definition
if it is not it is prime itself.
But because $\PP_{p_n}$ is bounded there are $x$ larger than $p_n$
that have prime factors larger than $p_n$ and can potentially not be factorized.
It might seem like no number larger than $p_n$ can be factorized by $\PP_{p_n}$
since any power of two can be factorized by any $\PP_{p_n}$.
In fact any number with $m_i \in \NN_0$ that has the form

$$2^{m_1} \cdot 3^{m_2} \cdot 5^{m_3} \cdot \ldots \cdot p_{n-1}^{m_{n-1}} \cdot p_n^{m_n}
= \prod_{i=1}^n p_i^{m_i} $$

is factorizable by $\PP_{p_n}$.
\st{The Sieve of Eratosthenes utilizes a similar principle
to remove all non prime numbers from its set.}
[Eratosthenes removes all multiples of any prime number of the set.
This means that also any higher primes larger than $p_n$ are removed.
But somehow (up to $p_n$?) all the numbers are met.
There might lie the core of the proof?!]

\marginpar{2025-01-24}


\subsubsection*{Attempt of some basic definitions}

\emph{Natural numbers} are from now on simply referred to as \emph{numbers}.
Every number not further specified is any natural number.

\emph{Prime numbers} are from now on simply referred to as \emph{primes}.

\emph{Prime factorizations} are from now on simply called \emph{factorizations}.

\emph{The set of all prime numbers} $\PP = \set{2, 3, 5, 7 \dots }$.

The convention to use from now on lowercase p only for prime numbers $p \in \PP$.

\emph{Factorization Set} is a set of consecutive prime numbers that can be utilized to factorize integers.

$$\PP_{x} \coloneq \Set{p}{p \leq x \in \PP}$$

$n \coloneq |\PP_x|$

$$\PP_{p_n}
= \set{2, 3, 5, \dots, p_n}
= \set{p_1, p_2, p_3, \dots, p_{n-1}, p_n}$$

The set of all natural numbers greater or equal than $2$ and less or equal than $m$ are called a \emph{Test Set}.

$$[m] \coloneq \set{2, 3, \dots, m}$$

The \emph{factorizability of a natural number} $x \in \NN \ge 2$ by a Factorization Set $\PP_{p_n}$
means that there is a (efficient) algorithm utilizing the numbers of this set to factorize the natural number.
The \emph{factorizability of a Test Set} $[m]$ means that that the Factorization Set has the ability to
factorize all numbers in the Test Set.
The Factorization Set $\PP_{p_n}$ has in general the ability to factorize multiple Test Sets;
the set off all this Test Sets is $\TT_{p_n}$.

$$\TT_{p_n} \coloneq \Set{t}{t\text{ is factorizable by }\PP_{p_n}}$$

\subsubsection*{Reasoning}

The question is now, what is the largest Test Set $[m^{p_n}_{max}]$ of $\TT_{p_n}$.
$\PP_{p_n}$ is obviously able to factorize $[p_n]$ since $\PP_{p_n}$ contains all prime in $[p_n]$.
If $x \in [p_n]$ is a prime, it will simply be factorized into itself $x \in \PP_{p_n}$.
If $x$ is not a prime, it will be factorized into multiple primes $p_i$.
Since the product of primes is greater than each of its factors,
each of those primes has to be smaller than $x$, therefore each $p_i \in \PP_{p_n}$.

\marginpar{2025-01-27}

The algorithm, outlined in the following steps, is referred to as the \emph{Factorization Algorithm}
or short the \emph{Algorithm}.

\begin{itemize}
 \item Try to divide consecutively by each prime $p_i \in \PP_{p_n}$.
 \item If the integer division is successful, try to repeat the division.
 \item If the integer division is not successful anymore, continue to the next $p_i$.
 \item In any case after a successful integer division the residual value is 1, the factorization is complete.
 \item If after the last $p_i$ (which is $p_n$) the residual value is still larger than 1,
 there are two options.
 If it can be determined that residual value is a prime, the factorization is complete.
 Else: it cannot be determined the residual value is prime, the factorization failed.
\end{itemize}

\marginpar{2025-01-29}

Every Factorization Set $\PP_x$ has at least 2 as Element.
Therefore every power of $2^m$ is factorizable by every Factorization Set,
since the number of repeated divisions required to factorize $2^m$ is $m$.
In the same way every product of powers of primes of a Factorization Set
is factorizable by this Factorization Set;
this is valid for any tuple of exponents $(m_1, \dots, m_n)$, with $m_i \in \NN_0$.

$$2^{m_1} \cdot 3^{m_2} \cdot 5^{m_3} \cdot \ldots \cdot p_{n-1}^{m_{n-1}} \cdot p_n^{m_n}
= \prod_{i=1}^n p_i^{m_i} $$

This means that there are numbers larger than $p_n$ that are factorizable by $\PP_{p_n}$.
But there are obviously other numbers larger than $p_n$ that are not factorizable by $\PP_{p_n}$,
like the product of two primes far larger than $p_n$ like $x \cdot y$ with $x,y \in \PP \gg p_n$.

For primes just larger than $p_n$ like $p_{n+1}$ or $p_{n+2}$ this is not necessary the case.
They cannot be represented by any $\prod p_i^{m_i}$;
but if it is for example deductible, that this number has to be a prime,
this number is automatically factorized (itself is its only factor).
In case the number is not prime, it has to be the product of at least two primes.
The product of two primes is larger than its factors.
$2 \cdot p_{n+1}$:
the smallest prime $2$ and the smallest prime larger than $p_n$ which is $p_{n+1}$.
This is the smallest two-factor number where one factor is larger than $p_n$.
Applying the Factorization Algorithm divides the number by two and the residual is $p_{n+1}$.
If the Algorithm could be sure, that this number is prime, it would be factorized.
Since it is known that the initial $2 \cdot p_{n+1}$ was the smallest two-factor number
not covered by $\prod p_i^{m_i}$, it cannot be a two-factor number outside of $\prod p_i^{m_i}$.

Let's make a proper definition:
$P_n$ is the set of all numbers that can be expressed as the product of primes in $\PP_{p_n}$

$$ P_n \coloneq \Set{x}{x = \prod_{i=1}^n p_i^{m_i}, p_i \in \PP_{p_n}, m_i \in \NN_0 }$$

...Since it is known that the initial $2 \cdot p_{n+1}$ was the smallest two-factor number
not in $P_n$, it cannot be a two-factor number not in  $P_n$.
The smallest $m$-factor number (with $m > 2$) is $2^{m-1} \cdot p_{n+1}$ is even larger than
the two-factor number.
Therefore $p_{n+1}$ has to be a one-factor number, which is per definition a prime number.

(Just for clarification: one could think that any number larger than $p_n$ could
be the product of other smaller primes.
But in that case the number would be in $P_n$ and would already have been factorized
by the Algorithm.
The trick is here, that $2 \cdot p_{n+1}$ is the smallest non-prime number not in $P_n$.)

Utilizing this method, every $m$-factor number with $m-1$ factors in $P_n$ could
be factorized.
In practice, given an arbitrary number, $m$ is unknown.
Therefore the residual of the Algorithm does not automatically have to be prime.
The question is now, what is the largest number (residual) that is guaranteed to be prime
given a certain Factorization Set.
$p_{n-1} \cdot p_n$, $p_n^2$ and $p_n \cdot p_{n+1}$ can still be factorized.
$p_{n+1}^2$ is not in $P_n$.
(By applying a square root, it could still be determined that $p_{n+1}$ is prime.
But it would hardly make any sense to implement a separate procedure just to cover
``a few'' extra numbers.)
$p_{n+1} \cdot p_{n+2}$ cannot be factorized anymore without ``extending'' $\PP_{p_n}$.

If this reasoning is correct, $p_{n+1}^2 - 1$ is the largest number
the Algorithm can be applied to and the residual to be safely assumed to be prime.
This allows the Factorization Algorithm to extend its ``range'' to more than the
square of the maximal element of its Factorization Set.

\subsubsection*{Proof?}

\marginpar{2025-02-06}

There might lie a proof by contradiction within this.

Theorem:
Given a Factorization Set $\PP_{p_n}$,
the smallest number,
that has more than one prime factor $x \not\in \PP_{p_n}$,
is $p_{n+1}^2$.

Proof:
Assuming there is a number $x \cdot y < p_{n+1}^2$,
that has two prime factors $x, y \not\in \PP_{p_n}$.
The smallest number having this prime factors is $x \cdot y$,
since a number with additional factors $p \in \PP_{p_n}$ is larger:
$x \cdot y \cdot p > x \cdot y$.
The smallest number $x \cdot y$ is formed by the smallest primes larger than $p_n$,
which are $x = p_{n+1}, y = p_{n+1}$.
Therefore the smallest $x \cdot y = p_{n+1} \cdot p_{n+1} = p_{n+1}^2$,
therefore every possible $x \cdot y \geq p_{n+1}^2$,
which is in contradicton to the initial assuption.
(None of the above statements is restricted by additional factors,
since adding additional factors would render the numbers only greater.)

\rightline{$\square$}


\marginpar{2025-02-07}


\subsection{Sieve of Eratosthenes}

A space saving way to implement the Sieve
is to utilize every bit of every number in an array,
since the index represents the numbers and the only information,
that has to be stored in the value is a boolean if the number is prime.
To improve the Sieve's time efficiency,
it can be prefilled with the pattern formed by the first prime numbers.

\begin{small}
\begin{verbatim}
       0    5    1    5    2    5    3    5    4    5    5    5    6
2:     x x x x x x x x x x x x x x x x x x x x x x x x x x x x x x x x
3:     x  x  x  x  x  x  x  x  x  x  x  x  x  x  x  x  x  x  x  x  x  x
5:     x    x    x    x    x    x    x    x    x    x    x    x    x
sum:   1 11111 111 1 111 1 111 11111 1 11111 111 1 111 1 111 11111 1 11
count: 1     5   3 1   3 1   3     5 1     5   3 1   3 1   3     5 1
                                     ^                             ^
                                     repeating pattern at 30 and 60
\end{verbatim}
\end{small}

The first prime numbers remove a substantial amount of the Sieve:
\begin{itemize}
 \item 2: removes $\frac{1}{2}$
 \item 3: removes an additional $\frac{1}{6}$
 \item 5: removes another additional $\frac{1}{15}$
\end{itemize}

Which sum up to $\frac{11}{15}$ or about 73\%.

On a 64 bit machine, choosing 64-bit integer might be advantageous.
Since the the pattern of the first three primes repeats on every 30\textsuperscript{th} number,
and 64 is not a multiple of 30,
the pattern hast to shift through the array indices.
The least common multiple is $960 = 64 \cdot 15 = 30 \cdot 32$;
therefore the pattern repeats on every 15\textsuperscript{th} 64-bit array entry.


\marginpar{2025-02-08}

The above pattern renders shows ones as hits of the Sieve for the given numbers.
The blank spaces represent candidates for new candidates.
To be more expressive in the implementation,
the pattern is reversed and candidates are interpreted at ones
and number excluded by the Sieve as zeros.
(The pattern shown above shows numbers starting from 0 up to 64.
Interpreting this directly would suggest to take 1 as the next prime number.
For the actual implementation the first prime numbers will be hard-coded and
the first ``page'' of the Sieve will not be used at all.)

To make the notation more compact, hexadecimal numbers are utilized.
A hexadecimal number represents 4 digits;
since 30 is not divisible by 4,
two patterns with combined 60 digits are used to generate a 15-digit hexadecimal pattern.
16 of this hexadecimal patterns consecutively represent 960 binary digits
and can simply be regrouped to 15 16-digit hexadecimal patterns.
Each of those 16-digit hexadecimal patterns represents a 64-bit number.


\begin{footnotesize}
\begin{flalign*}
&\neg101111101110101110101110111110 = 010000010001010001010001000001\\[4pt]
&010000010001010001010001000001\;010000010001010001010001000001_2 = 411451050451441_{16}\\[4pt]
&411451050451441\;411451050451441\;411451050451441\;411451050451441\\
&411451050451441\;411451050451441\;411451050451441\;411451050451441\\
&411451050451441\;411451050451441\;411451050451441\;411451050451441\\
&411451050451441\;411451050451441\;411451050451441\;411451050451441_{16} =\\[4pt]
&4114510504514414\;1145105045144141\;1451050451441411\;\\
&4510504514414114\;5105045144141145\;1050451441411451\;\\
&0504514414114510\;5045144141145105\;0451441411451050\;\\
&4514414114510504\;5144141145105045\;1441411451050451\;\\
&4414114510504514\;4141145105045144\;1411451050451441_{16}\\
\end{flalign*}
\end{footnotesize}

\marginpar{2025-03-02}

Another idea is to reduce the array by leaving out all the numbers
that cannot be primes according to the former scheme.

\begin{small}
\begin{verbatim}
          1         2         3         4         5         6
0123456789012345678901234567890123456789012345678901234567890
101111101110101110101110111110101111101110101110101110111110
 0     0   0 0   0 0   0     0 0     0   0 0   0 0   0     0
 1     7   1113  1719  23    2931    37  4143  4749  53    59

1  7  11 13 17 19 23 29 31 37 41 43 47 49 53 59
A     AA                A     AA                  (0 2) ( 1 11)
   B        BB             B        BB            (1 4) ( 7 17)
         C        CC             C        CC      (3 6) (13 23)
               D     DD                D     DD   (5 7) (19 29)
0  1  2  3  4  5  6  7  8  9  10 11 12 13 14 15

exp    cont
0  1    1 0
1  7    7 1
2 11   11 2
3 13   13 3
4 17   17 4
5 19   19 5
6 23   23 6
7 29   29 7
\end{verbatim}
\end{small}

One (potentially small) advantage of this concept would be,
that (for the 2,3,5-scheme) the repeating pattern of 30 numbers
results in 8 potential prime numbers and therefore would fit into a byte.

The concrete period of 30 is from now on generalized to $T$;
the scheme to $S_x$ with the 2,3,5-scheme being $S_5$.
$[x] \coloneq \set{0,1,\dots,x}$.

If $C_5$ is the set of all candidates of the 2,3,5-scheme,
a contraction function $c_5(x): C_5 \rightarrow \NN$
and a expansion function $e_5(x): \NN \rightarrow C_5$ would be required.
It can be expected that the lower memory requirement would speed up the
execution of the Sieve.
At this point it is unclear if the application of those functions
would negate the speed gain from the lower memory requirement.

The contraction function is mainly required for the application of the Sieve,
eliminating candidates as multiples of prime numbers.
To concretize this, the candidate set $C_s^T$ of a scheme $s$ is limited to a period $T$;
the corresponding set of natural numbers is then $[|C_s^T|]$.
$$c_s^T(x): C_s^T \longrightarrow [|C_s^T|]$$
$$e_s^T(x): [|C_s^T|] \longrightarrow C_s^T$$

\begin{flalign*}
c_5^{30}(x): C_5^{30} &\longrightarrow [7]    \\
        1 &\longmapsto 0\\
        7 &\longmapsto 1\\
          &\cdots\\
        29 &\longmapsto 7
\end{flalign*}

\begin{flalign*}
e_5^{30}(x): [7] &\longrightarrow C_5^{30}\\
        0 &\longmapsto 1\\
        1 &\longmapsto 7\\
          &\cdots\\
        7 &\longmapsto 29
\end{flalign*}

The algorithm might have the following main steps,
given a scheme $s$ and a period $T$ starting with the number $x$.
All bits of the array are initialized with ones as potential candidates;
candidates are eliminated by marking them with zero:

\begin{itemize}
 \item Splitting $x$ into a offset $o = (x\mod T)$ and a base $b = x - o$
 \item If $o \not \in C_s^T$ stop the algorithm; $x$ is not prime
 \item Generate a bit mask by calculating $2^{c(o)}$
 \item If $x$ is smaller than $T$ repeat the former step accordingly and
 reduce the result with bitwise OR operations
 \item Negate the bit mask to allow elimination
 \item \st{Loop, with $i \geq 2$ as the iterator, over every $ib$\textsuperscript{th} element of the array}
 \item \st{Apply the negated bit mask by utilizing bitwise ADD to the array element}
\end{itemize}
[The splitting cannot be done beforehand, since the numbers are not multiples of $T$.]

\section{Greatest Common Divisor}
\marginpar{2025-03-24}

I just found out there is an simple and efficient algorithm for finding the $gdc(n)$
called Euclidean Algorithm.
The Euclidean Algorithm—and it's less efficient companion called Euclid's Algorithm—require
no prime numbers.
Hence the prime number investigation will stop for now.
For $a, b \in \NN$ it is:

\begin{flalign*}
 gcd(a, b) = \left\{
  \begin{aligned}
   &a &b = 0\\
   &gcd(b, \bmod(a, b)) &b\not= 0
  \end{aligned}
 \right.
\end{flalign*}

Note that for $a > b \Rightarrow a > b > \bmod(a, b)$ and
for $a < b \Rightarrow \bmod(a, b) = a$ (which switches $a$ and $b$) and therefore
the first relation holds for every consecutive function call.
Edge cases are: $gcd(a, 0) = a$, $gcd(0, b) = b$ and $gcd(0, 0) = 0$
which is acceptable, since in praxi $0/0$ means 0 for most cases.


\end{document}
