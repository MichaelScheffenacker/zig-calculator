\documentclass[a4paper,10pt]{article}
\usepackage[utf8]{inputenc}
\usepackage{amsfonts}
\usepackage{mathtools}

%opening
\title{Calculator Notes}
\author{msc}

\begin{document}

% \maketitle

% \begin{abstract}

% \end{abstract}

% \section{}

\begin{verbatim}
 $ 1 / 2
 1
 -
 2
\end{verbatim}

$$\frac{1}{2}$$


\begin{verbatim}
 $ 1 / 2 / 3
  1
  -
  2
 ---
  3
\end{verbatim}

\[\frac{\frac{1}{2}}{3}\]


\begin{verbatim}
 $ 1 / 2 / 3 / 4
   1
   -
   2
  ---
   3
 -----
   4
\end{verbatim}
\[
\frac{\frac{\frac{1}{2}}{3}}{4}=\frac{\frac{1}{2\cdot3}}{4} = \frac{1}{2\cdot3\cdot4}
\]


\begin{verbatim}
 $ 1 / 2 / (3 / 4)
\end{verbatim}
\[
\frac{\frac{1}{2}}{\frac{3}{4}}=\frac{1\cdot4}{2\cdot3}
\]


\begin{verbatim}
 $ 5 * 6 * 1 / 2 / 3 / 4
\end{verbatim}
\[
5\cdot6\cdot\frac{\frac{\frac{1}{2}}{3}}{4} =
5\cdot6\cdot\frac{\frac{1}{2\cdot3}}{4} =
5\cdot6\cdot\frac{1}{2\cdot3\cdot4} =
30\cdot\frac{1}{24} =
\frac{30}{24} =
\frac{15}{12} =
\frac{5}{4}
\]


\begin{verbatim}
 $ 1 / 2 / 3 / 4 * 5 * 6
\end{verbatim}
\[
\frac{\frac{\frac{1}{2}}{3}}{4}\cdot5\cdot6 =
\frac{\frac{1}{2\cdot3}}{4}\cdot5\cdot6 =
\frac{1}{2\cdot3\cdot4}\cdot5\cdot6 =
\frac{1}{24}\cdot30 =
\frac{30}{24} =
\frac{15}{12} =
\frac{5}{4}
\]


\begin{verbatim}
 $ 1 / 2 * 5 / 3 * 6 / 4
\end{verbatim}
\[
\frac{1}{2}\cdot\frac{5}{3}\cdot\frac{6}{4} =
\frac{1}{2}\cdot\frac{5}{3}\cdot\frac{3}{2} =
\frac{1\cdot5}{2\cdot3}\cdot\frac{6}{4} =
\frac{5}{6}\cdot\frac{6}{4} =
\frac{5\cdot6}{6\cdot4} =
\frac{30}{24} =
\frac{5}{4}
\]


\begin{verbatim}
 $ 1 / 2 / 3 / 4 + 5 / 6
   1
   -
   2
  ---
   3     5
 ----- + -
   4     6
\end{verbatim}


\begin{verbatim}
 $ 1 / 2
   1
   -
   2
\end{verbatim}


\begin{verbatim}
 $ 1 / 2 / 3 / 4
     1
   -----
   2·3·4
\end{verbatim}


\begin{verbatim}
 $ 1 / 2 / 3 / 4 * 5 * 6
   1·5·6
   -----
   2·3·4
\end{verbatim}



\section{Prime Factorization}

Brute force calculation of primes failed;
even the first $10^5$ take 10\,s here on Witt\-gen\-stein.

Proper prime factorization algorithms that factorize numbers above $10^{100}$ would at least require to look into number theory and algebra.
Apparently implementation of such algorithms took something in the order of years rather than months.

A better idea might be to implement a prime sieve to calculate a sufficient number of primes and then brute force divide through the numbers.


\subsection{Sieve of Eratosthenes}
This is the simplest sieve which just removes all multiples of each prime number (within the array).
A simple and efficient lookup might be an array with all numbers up to the upper bound.
A reasonable upper bound might be $2^{33} = 8\,589\,934\,592$ .
This should safely allow to factorize all numbers up to $(2^{33})^2 = 2^{66} = 73\,786\,976\,294\,838\,206\,464$
[better check by investigating the largest primes smaller than this numbers].

Utilizing an array with the full range of numbers allows to identify the offset as the value of the number,
therefore the array's value can be used as boolean prime indicator.

\begin{verbatim}
0 1 2 3 4 5 6 7 8 9
0 0 1 1 1 1 1 1 1 1
    ^ first prime: 2

0 1 2 3 4 5 6 7 8 9
0 0 1 1 0 1 0 1 0 1
        ^   ^   ^ multiples of 2

0 1 2 3 4 5 6 7 8 9
0 0 1 1 0 1 0 1 0 1
      ^ prime: 3

0 1 2 3 4 5 6 7 8 9 0 1 2 3 4 5
0 0 1 1 0 1 0 1 0 0 0 1 0 1 0 0
            ^     ^     ^     ^ multiples of 3

0 1 2 3 4 5 6 7 8 9 0 1 2 3 4 5
0 0 1 1 0 1 0 1 0 0 0 1 0 1 0 0
          ^ prime: 5

...
\end{verbatim}

[apparently the smallest untouched multiple of a new prime is its square, might want to include that ...]

More ideas:
\begin{itemize}
 \item Condensate eight Booleans into a byte; would reduce memory usage to an eighth. To evaluate the overhead of extracting the values would require testing.
 \begin{itemize}
  \item Lower numbers would require to bitwise-and their primality out of the byte.
  \item In the range of larger numbers, prime numbers get more sparse.
  The last prime number before $2^{33}$ is 8\,589\,934\,583.
  According to Wolfram Alpha 8\,589\,934\,583 is the 393\,615\,806\textsuperscript{th} prime number.
  This means on average up to this prime number about every 22\textsuperscript{nd} number is prime.
  So even if the prime numbers were evenly spread over this range almost two out of three bytes would be zero,
  allowing a simple, initial zero check before having to bitwise-and the number.
  (If this this actually more efficient is questionable since the bitwise and should be cheap anyways
  while transferring the data from memory might be the expensive operation.)
  \item Another idea:
  If most of the bytes are zero anyways there might be a relatively simple function that identifies the offsets of most of the zero bytes.
  In this case running this function from cache might be faster than fetching all this data from main memory.
  On the other hand the prefetcher might nullify this advantage.
 \end{itemize}
 \item Skip every second value.
 Since, except for two, even numbers are not prime anyways,
 skipping every even number should half calculation time and memory space.

\end{itemize}


\subsection*{2024-09-16}

Numbers up to $2^{33}$ obviously requires 64 bit unsigned integers;
to save space it might be sufficient to utilize prime numbers up to $2^{32}$.
The largest prime number smaller than $2^{32}$ is $2^{32} - 5 = 4\,294\,967\,291$
which is the 203\,280\,221\textsuperscript{st} prime number.
Therefore an array holding all primes numbers up to $2^{32} - 5$ as unsigned 32 bit integers would require around 800\,MB.
The closest prime number below $(2^{32} - 5)^2$ is $(2^{32} - 5)^2 - 2 = 18\,446\,744\,030\,759\,878\,679$.
But integers greater than this number can still be factorized by utilizing $2^{32} - 5$
since the division of $(2^{32} - 5)^2 - 2$ will fail
and therefore render it prime.
But for larger integers divisions will succeed
since they are not prime and therefore prime numbers up to $2^{32} - 5$ will factorize them.
The first number this will fail is the first prime number larger than $(2^{32} - 5)^2$
which is $(2^{32} - 5)^2 + 40 = 18\,446\,744\,030\,759\,878\,721$.
Therefore every integer up to 18\,446\,744\,030\,759\,878\,720 can be factorized by the set of all prime numbers up to $2^{32} - 5$.

\subsection*{2024-09-19}
Okay, we need some nomenclature here:
\begin{itemize}
 \item $B_{32} := 2^{32} = 4\,294\,967\,296$ one above the largest \texttt{unsigned integer}
 \item $B^-_{32} := B_{32} - 5 = 2^{32} - 5 = 4\,294\,967\,291$ the largest prime number smaller than $B_{32}$ (203\,280\,221\textsuperscript{st} prime number)
 \item $B^+_{32} := B_{32} + 15 = 2^{32} + 15 = 4\,294\,967\,311$ the smallest prime number larger than $B_{32}$ (203\,280\,222\textsuperscript{nd} prime number)
 \item $B_{\dot{64}} := {B^-_{32}}^2 = (2^{32} - 5)^2 = $
 \item $B^-_{\dot{64}} := B_{\dot{64}} - 2 = (2^{32} - 5)^2 - 2 = 18\,446\,744\,030\,759\,878\,679$
 \item $B^+_{\dot{64}} := B_{\dot{64}} + 40 = (2^{32} - 5)^2 + 40 = 18\,446\,744\,030\,759\,878\,721$
\end{itemize}


There is an error in the considerations above:
Primes numbers larger than $B_{\dot{64}}$ can actually be detected
since finding no prime divisor up to $B^-_{32}$ simply means the number in question is prime;
this is true up to the smallest number that has two prime factors larger than $B^-_{32}$.
(For example this is not true for $B^-_{32}\cdot B^+_{32}$
since $B^-_{32}$ is in the \texttt{prime number array}
and after the division the result is $B^+_{32}$ which turns out to be prime and the factorization is complete.)
The smallest number composed of the smallest two prime numbers larger than $B^-_{32}$ is ${B^+_{32}}^2 = 18\,446\,744\,202\,558\,570\,721$.
${B^+_{32}}^2$ is 171\,798\,692\,000 larger than $B^+_{\dot{64}}$,
the former candidate for the upper limit of the prime factorization with factors up to $B^-_{32}$.

\subsection*{2024-09-23}

\newcommand{\PP}{\mathbb{P}}
\newcommand{\Bm}[1]{B^-_{32}}
\newcommand{\Bp}[1]{B^+_{32}}
\newcommand{\set}[1]{\lbrace \, {#1} \, \rbrace}
\newcommand{\Set}[2]{\set{ {#1} \;|\; {#2} }}

Let's try to bring this into a more formal form.
$\PP$ the Set of all prime numbers.

$$\PP_{\Bm{32}} = \Set{p}{p \in \PP \leq \Bm{32}}$$

$$\PP_{\Bp{32}} = \Set{p}{p \in \PP \leq \Bp{32}} = \set{2, 3, 5, \dots, \Bm{32}, \Bp{32}}$$

$$\PP_{x} \coloneq \Set{p}{p \in \PP \leq x \in \PP}$$

$n \coloneq |\PP_x|$; $p_i \in \PP$

$$\PP_{p_n}
= \set{2, 3, 5, \dots, p_n}
= \set{p_1, p_2, p_3, \dots, p_{n-1}, p_n}$$

\end{document}
